In this thesis, we have applied supervised machine learning algorithms to simulated and experimental
nuclear physics data. With the goal of leveraging the exact knowledge of properties in simulations to extract
vital patterns and representations, we trained models to classify simulated decay events as either 'single'
or 'double', and to predict the energy and positions of origin for events. To determine the effects of
some properties of the experimental data, we prepared additional simulated datasets where we incorporated
these properties. They were so-called 'dead' pixels in specific locations, and imbalanced class distribution.
We found that for simulated data a Convolutional Neural Network-based architecture is highly successful in the
classification of events across all the prepared datasets, with $F1$ scores up to $0.97$. 
This is not the case for the 'simpler' logistic and dense models, which suffer from the included properties 
in the datasets.
Classification on experimental data was not found to be successful, regardless of model performance on
simulated data. This may be partly explained by a difference in total intensity, or deposited energy,
between simulated and experimental data. We observe a trend across all models and training data, that
when classifying experimental decay events, most all event classified as single are in the lower
regime of total intensities. We observe the same trend if we classify simulated data where all pixel
intensities are scaled up.

We successfully trained models to predict positions of origin for simulated single decay events,
with $R2$ scores of $0.99$ and above for all simulated datasets. However, we were unable to
predict positions of origin for simulated double decay events with any degree of precision.
We found the same problem when predicting energies, and energy prediction was also sensitive
to the modifications in the data. A likely cause of this sensitivity may be the strong correlation
between energy and total pixel intensity in detector images, and the modifications effectively
acting as removal of information.

Without true positions for experimental data to directly test the models against,
making any strong conclusions for these results is hard. Proper verification of performance 
will have to wait until researchers can test the predicted positions on source data from the
experiment. However, we find that the positions predicted seem within reasonable limits, compared
with what is seen in simulated data. There is a risk that when the number of 'dead' pixels in a detector
image increases, the positions predicted may not be positions of the event itself, rather a position
based on the remaining information in the image.

We found similar results when predicting energies for experimental decays. Preliminary comparisons
with experimental calibration constants indicate that the predictions are not good on experimental
decays. The working theory is that the simulated training data predicts too many photons per unit
energy, resulting in the knowledge from simulated data not being transferable to the experimental
domain.

\section{Future work}
From the results found in this thesis, there may be grounds to adjust simulations based on
experimental data, or post-process the simulated datasets to better represent the experiment.
Another option could be to invest in hand labelling an experimental dataset which can be used
to tune the models used in this thesis. This could also allow for estimating upper and lower bounds
for the Receiver Operating Characteristic \ref{Claesen2015}, which we have not used in this thesis.
As the effect of pixels being set to zero value is detrimental to energy predictions, in particular,
efforts might be made to assess the magnitute of the effect, and whether or not it could be reasonable
to approximate pixel values.