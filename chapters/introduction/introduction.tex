\noindent In this thesis, we the gap between simulation and experiment.
Modelling the world around us to predict behaviour and increase our understanding is a key concept
in physics, and science as a whole. Simulating physical processes often lay the groundwork for verifying
theory, and designing experiments. For many branches of physics, and especially nuclear physics,
there has been an exponential growth in data generated from experiments. This brings several challenges,
the most immediate of which may be "how do we process it all?". If there are multiple types of
reactions occurring in an experiment, how can we separate them when there are billions of events?
In the later years, machine learning algorithms have gained popularity for accomplishing this task,
alleviating the need for humans to sift through vast amounts of data by hand. In this thesis, we
explore the leveraging of simulated experiments to train machine learning models and subsequently
apply them to data gathered from real-world experiments. 



Nuclear physics seeks to understand the building blocks of the universe - nuclei - and from that 
understanding build a comprehensive and predictive model of them. The vast majority of nuclei discovered 
so far are not stable, and the study of many of them requires specialized equipment.


