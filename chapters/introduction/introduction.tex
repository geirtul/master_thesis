\noindent In this thesis, we approach the problem arising from copious amounts of unlabeled data
generated in nuclear physics experiments. Modelling the world around us to predict behaviour and increase 
our understanding is a key concept in physics, and science as a whole. Simulating physical processes often 
lay the groundwork for verifying theory, and designing experiments. For many branches of physics, 
and especially nuclear physics, there has been an exponential growth in data generated from experiments. 
This brings several challenges, the most immediate of which may be "how do we process it all?". If there are 
multiple types of
reactions occurring in an experiment, how can we separate them when there are billions of events?
In the later years, machine learning algorithms have gained popularity for accomplishing this task,
alleviating the need for humans to sift through vast amounts of data by hand. 
Machine learning has already been used in a wide variety of applications, such as at CERN \cite{Arpaia2021},
and the Active-Target Time Projection Chamber (AT-TPC) at the National Superconducting Cyclotron Laboratory
at Michigan State University \cite{Kuchera2019}. In this thesis, we explore the leveraging of simulated experiments to train machine learning models and subsequently
apply them to data gathered from real-world experiments.


\noindent In simulations, we have complete knowledge of every parameter. The energy of a decay, the exact positions
of the ion where the decay originated, the exact time of the decay are all examples of knowledge we have about
simulations. For experimental data, these details are what we seek to learn in order to verify and improve our
model of the physical universe, and they are not necessarily directly measurable.
Traditional methods of analysis may be computationally expensive to run, especially considering the
amounts of data these methods must churn through, with repeated calculations for every sample.
With machine learning algorithms we can front-load all these computations, training them to perform
the task of extracting our desired information based on simulations. If our simulations capture crucial
attributes of the real data, we may be able to approximate even values that would be notoriously difficult 
to measure or infer from data alone. Another avenue is determining the types of reactions or events that
occur in experimental data. This is of particular interest in this specific case, as there is currently no methods
other than the human eye to separate events into their respective categories. 

\noindent We investigate the possibility of carrying out three different tasks. Classification, prediction of event energy, and prediction of
positions of origin. We introduce the fundamental concepts for machine learning theory, our methods and
implementation, and present results of carrying our these tasks on simulated and experimental data.