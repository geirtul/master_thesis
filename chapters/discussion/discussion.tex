\section{Classification}
With the goal of classifying experimental decay data, we set a minimum requirement
of first showing that machine learning algorithms can classify simulated decay events.
From the $F1$ scores shown in table \ref{tab:classification-simulated-all-f1-auc} we
can see that classification of simulated events is possible.
There is a clear performance gap between CNN-based architectures and the logistic regressor and dense
neural network. This gap may be expected, as CNN's are specifically designed for
machine learning involving images. Of the CNN architechtures, the deeper architectures
(Custom \ref{appendix:model-custom}, Pretrained \ref{appendix:model-pretrained}),
show less decrease in performance when trained on datasets with 'dead' pixels and
imbalanced representation of classes. When training models on an imbalanced dataset
these deeper models also appear less prone to simply classify most of events as one
class. In figure \ref{fig:confmat-simulated} we can see that this is the case for
the logistic and dense models.

With the ability to classify simulated events as either single or double decays,
we then applied the models trained on simulated data to experimental data.
Lacking true labels for experimental data, we look to other expecations, such
as the fraction of events predicted as single and double decays, shown in table
\ref{tab:classification-experimental-ratios}. Our expectation is that there is
a much larger amount of single decays present in experimental data than
double decays. In light of this expectation, the models' performance is not very good.
The models performing best on simulated data predict up to 90\% or more of the
experimental decays to be double decays. This is also the case for models trained
on dataset $c$ (imbalanced). In figure \ref{fig:comparison-intensity} we saw that
there is a difference in total intensity between simulated decays and experimental
decays. The experimental decays range higher in total intensity, and in figure
\ref{fig:intensity-hip-comparison} we also see that a higher maximum intensity in
an experimental event corresponds to a higher total intensity than for simulated
events. Additionally, the fraction of predicted single events as a function of total
intensity in images which we show in figure \ref{fig:experimental-single-fractions},
indicate that most single events are predicted for low total intensities in images.
In fact, single events are predicted almost exclusively in the region of total intensity
where single events are distributed in simulated data. We also concider the trend
of decreased fraction of correctly classified single decays in simulated data,
presented in figure \ref{fig:simulated-scaled-intensities}. Together,
this difference between simulated and experimental data may partially explain why a 
lot of events are classified as double decays.

\section{Regression}
\subsection{Positions of origin}
\subsection{Energy}
With energies being closely correlated with total intensity in an image,
having dead pixels can be detrimental to the performance in predictions.
- show this somehow?